\documentclass[10pt,a4paper,french]{article}
\usepackage[T1]{fontenc}
\usepackage[utf8]{inputenc}
\usepackage{fourier}
\usepackage{xspace}
\usepackage[scaled=0.875]{helvet}
\renewcommand{\ttdefault}{lmtt}
\usepackage{amsmath,amssymb,amstext,makeidx}
\usepackage{fancybox}
\usepackage{tabularx}
\usepackage[normalem]{ulem}
\usepackage{pifont}
\usepackage[euler]{textgreek}
\usepackage{textcomp,enumitem}
\usepackage[table]{xcolor}
\usepackage{lscape,rotating}
\usepackage{graphicx}
%tapuscrit Jean-Claude Souque
\newcommand{\euro}{\eurologo{}}
\usepackage{pst-tree,pst-plot,pst-text,pst-eucl,pst-bar,pst-func,pst-math,pstricks-add}
\usepackage[squaren, Gray, cdot]{SIunits}
\newcommand{\R}{\mathbb{R}}
\newcommand{\N}{\mathbb{N}}
\newcommand{\D}{\mathbb{D}}
\newcommand{\Z}{\mathbb{Z}}
\newcommand{\Q}{\mathbb{Q}}
\newcommand{\C}{\mathbb{C}}
\usepackage[left=3.5cm, right=3.5cm, top=1.9cm, bottom=2.7cm]{geometry}
\newcommand{\vect}[1]{\overrightarrow{\,\mathstrut#1\,}}
\renewcommand{\theenumi}{\textbf{\arabic{enumi}}}
\renewcommand{\labelenumi}{\textbf{\theenumi.}}
\renewcommand{\theenumii}{\textbf{\alph{enumii}}}
\renewcommand{\labelenumii}{\textbf{\theenumii.}}
\def\Oij{$\left(\text{O}~;~\vect{\imath},~\vect{\jmath}\right)$}
\def\Oijk{$\left(\text{O}~;~\vect{\imath},~\vect{\jmath},~\vect{k}\right)$}
\def\Ouv{$\left(\text{O}~;~\vect{u},~\vect{v}\right)$}
\usepackage{fancyhdr}
\makeatletter
\def\hlinewd#1{%
\noalign{\ifnum0=`}\fi\hrule \@height #1 %
\futurelet\reserved@a\@xhline}
\makeatother

\usepackage[dvips]{hyperref}
\hypersetup{%
pdfauthor = {APMEP},
pdfsubject = {Baccalauréat STL biotechnologies},
pdftitle = {Métropole - 6 septembre 2018},
allbordercolors = white,
pdfstartview=FitH}

\usepackage{babel}
\usepackage[np]{numprint}
\renewcommand\arraystretch{1.2}
\newcommand{\e}{\text{e}}
\frenchbsetup{StandardLists=true}
\begin{document}
\setlength\parindent{0mm}

\rhead{\textbf{A. P{}. M. E. P{}.}}
\lhead{\small Baccalauréat  Sciences et Technologies de laboratoire (STL)spécialité Biotechnologies}
\lfoot{\small{métropole}}
\rfoot{\small{6 septembre 2018}}
\pagestyle{fancy}
\thispagestyle{empty}
\begin{center}{\Large \textbf{\decofourleft~Baccalauréat STL spécialité biotechnologies  Métropole~\decofourright\\[5pt] 6 septembre  2018}}
\end{center}

\vspace{0,25cm}
 
\textbf{\textsc{Exercice 1} \hfill (6 points)}  

\vspace{0.25cm}

\emph{Dans cet exercice, les résultats seront arrondis à l’unité, sauf mention contraire.}

\medskip

Cyprien et Cloé, élèves de Terminale STL Biotechnologies, s’intéressent au protocole d’un examen
d’imagerie médicale, qui nécessite de préparer une dose de produit radioactif. Ils disposent du tableau
suivant, qui donne le nombre de milliards de noyaux radioactifs présents dans le produit préparé en
fonction du temps $t$, exprimé en minutes.

\smallskip
\begin{center}

\begin{tabularx}{\linewidth}{|m{4cm}|*{7}{>{\centering \arraybackslash}X|}}
\hline
Temps en minutes : $t_i$ 		&0 			&20		& 40		& 60	& 80		& 100		& 120\\\hline
Nombre de milliards
de noyaux radioactifs : $y_i$	&\np{8000}&\np{7400}&\np{6800} & \np{6300}&\np{5800} &\np{5400} & \np{4900}\\\hline
\end{tabularx}

\end{center}
\smallskip 

Au moment de son injection au patient, le produit ne doit pas contenir plus de \np{2600} milliards de noyaux
radioactifs. Cyprien et Cloé souhaitent alors déterminer combien de temps, une fois le produit prêt, il
faut attendre avant de l’injecter au patient.
\medskip

\textbf{Partie A}

\medskip

Cyprien propose d’utiliser un ajustement affine.

\medskip

\begin{enumerate}
\item  Représenter le nuage de points de coordonnées $(t_i , y_i)$ dans le repère fourni en \textbf{annexe, à rendre avec la copie}. Dans ce repère, 1 unité est égale à 20 minutes en abscisse et à 200 milliards de noyaux
radioactifs en ordonnée.
\item Déterminer, à l’aide de la calculatrice, une équation de la droite d’ajustement de $y$ en $t$ par la
méthode des moindres carrés. On la mettra sous la forme $y = at + b$, les réels $a$ et $b$ étant arrondis à $0,1$.
\item Cyprien admet que pendant les quatre heures suivant la préparation du produit, le nombre $y$ de
milliards de noyaux radioactifs encore présents dans le produit peut être modélisé par
$y= -26 t + \np{7900}$ où $t$ est le temps, exprimé en minutes, écoulé depuis que le produit a été préparé.
\begin{enumerate}
\item Tracer la droite $d$ d’équation $y = -26 t + \np{7 900}$ sur le graphique précédent.
\item Cyprien utilise alors ce modèle pour déterminer, combien de temps, une fois le produit prêt, il
faut attendre avant de l’injecter au patient. Quel temps Cyprien trouve-t-il ?
\item Une nouvelle mesure est fournie à Cyprien : au bout de 240 minutes, le nombre de milliards de
noyaux radioactifs présents dans le produit préparé est égal à \np{3100}. Cette mesure remet-elle en
cause le modèle utilisé par Cyprien ? Pourquoi ?
\end{enumerate}
\end{enumerate}

\medskip

\textbf{Partie B}

\medskip

Cloé remet en cause le modèle utilisé par Cyprien. L’examen exploitant un produit radioactif,  elle
préfère utiliser un ajustement exponentiel : le nombre $y$ de milliards de noyaux radioactifs encore
présents dans le produit peut être modélisé par  $y= A_0\e^{-k t} $ où $t$ est le temps, exprimé en minutes, écoulé
depuis que le produit a été préparé, $A_0$ et $k$ étant deux réels strictement positifs.

\medskip

\begin{enumerate}
\item Proposer une démarche qui permette de déterminer des valeurs prises par les réels $A_0$ et $k$.
\item Dans ce qui suit, on prend $A_0 = \np{8000}$ et $k = \np{0,0039}$.

Cloé utilise alors ce modèle pour déterminer, combien de temps, une fois le produit prêt, il faut
attendre avant de l’injecter au patient. Quel temps Cloé trouve-t-elle ?
\end{enumerate}

\vspace{0,25cm}

\textbf{\textsc{Exercice 2} \hfill (7 points)}  

\vspace{0.25cm}

\emph{Les deux parties de cet exercice peuvent se traiter de façon indépendante.}

\medskip

\textbf{Partie A}

\medskip

En 2017, une étude menée dans une ville a montré que la consommation d’eau par an et par habitant
s’élevait à \np[m^3]{50}. On suppose que, dans les années qui suivront, cette consommation baissera
de 2,1\,\% par an.

\medskip

\begin{enumerate}
\item  On considère l’algorithme suivant :
\begin{center}
\begin{tabular}[]{|m{1cm} m{3cm}|}
\hline
&$n\leftarrow 0$\\
&$u\leftarrow 50$\\
&Tant que $u > 47$\\
&$u\leftarrow  u \times 0,979$\\
&$n \leftarrow n+1$\\
&Fin Tant que\\
&$n\leftarrow n + 2017$\\
\hline
\end{tabular}
\end{center}

	\begin{enumerate}
		\item Quelle est la valeur de la variable $n$ à la fin de l’exécution de l’algorithme ?
		\item Interpréter ce résultat en termes de consommation d’eau par an et par habitant dans cette ville.
	\end{enumerate}
\item On modélise la situation par une suite ($u_n)$ où $u_n$ représente la consommation d’eau, en m$^3$, par an et par habitant, dans cette ville en $2017 + n$.
	\begin{enumerate}
		\item Montrer que la suite $\left(u_n\right)$ est géométrique. Préciser son premier terme et sa raison.
		\item Exprimer, pour tout entier naturel $n$, $u_n$ en fonction de $n$. 
		\item Avec ce modèle, quelle consommation d’eau peut-on prévoir par an et par habitant en 2021 ?

On arrondira à \np[m^3]{0,1}.
\end{enumerate}
\item On s’intéresse à la consommation totale d’eau par habitant depuis le début de l’année 2017.

Déterminer à partir de quelle année cette consommation totale d’eau dépassera \np[m^3]{500}.
\end{enumerate}

\medskip

\textbf{Partie B}

\medskip

\emph{Dans cette partie, les probabilités demandées seront arrondies à $10^{-4}$.}

\medskip

On sait qu’en 2017, 37\,\% des logements de la ville étaient équipés de systèmes de réduction de la
consommation d’eau (robinets mousseurs, récupérateurs d’eau de pluie, $\dots$).

\medskip

\begin{enumerate}
\item On considère un échantillon de \np{1 000} logements pris au hasard parmi les logements de la ville,
suffisamment nombreux pour assimiler le choix de cet échantillon à un tirage avec remise. On note
$X$ la variable aléatoire égale au nombre de logements de l’échantillon qui étaient équipés de
systèmes de réduction de la consommation d’eau en 2017.
	\begin{enumerate}
		\item Déterminer la loi suivie par la variable aléatoire X. Préciser les paramètres de cette loi.
		\item Quelle est la probabilité qu’il y ait plus de 400 logements dans l’échantillon qui étaient
équipés de tels systèmes de réduction de la consommation d’eau en 2017 ?
	\end{enumerate}
\item On décide d’approcher la variable aléatoire $X$ par une variable aléatoire $Y$ qui suit la loi normale de
paramètres \textmu = 370 et \textsigma = 15,27.
	\begin{enumerate}
		\item Justifier les valeurs choisies pour \textmu{} et \textsigma.
		\item Calculer la probabilité $P(Y \leqslant 350)$. Interpréter le résultat obtenu dans le contexte.
	\end{enumerate}
\end{enumerate}

\vspace{0,25cm}

\textbf{\textsc{Exercice 3} \hfill (7 points)}  

\vspace{0.25cm}

\emph{Les deux parties de cet exercice sont indépendantes.}

\medskip

Un antibiotique est une substance chimique organique inhibant ou tuant des bactéries pathogènes.

\medskip

\textbf{Partie A}

\medskip

Un laboratoire affirme que 48\,\% de toutes les souches bactériennes sont résistantes aux antibiotiques.

Dans un échantillon de 50 souches bactériennes prises au hasard, on constate que 29 souches sont
résistantes. Cela remet-il en cause l’affirmation du laboratoire ? Justifier.

\medskip

\textbf{Partie B}

\medskip

On injecte un antibiotique à un patient. On modélise cette situation par une fonction $f$ qui, à tout temps $t$,
exprimé en heures, écoulé depuis l’injection, associe la concentration, exprimée en mg$\cdot \text{L}^{-1}$, de
l’antibiotique dans le sang du patient.

Cette fonction $f$ est définie sur l’intervalle $[0, +\infty[$ par 

\[f (t)=\dfrac{8t}{t^2+1}.\]

\begin{enumerate}
\item  On admet que la limite de la fonction $f$ en $+\infty$ est égale à 0.
	\begin{enumerate}
		\item Interpréter la valeur de la limite pour la courbe représentative de la fonction $f$.
		\item Interpréter la valeur de la limite dans le contexte de l’exercice.
	\end{enumerate}
\item On note $f'$ la fonction dérivée de la fonction $f$.
	\begin{enumerate}
		\item On admet que pour tout réel $t$ positif ou nul, on a :$f'(t)=\dfrac{8(1- t)(1+ t)}{(t^2+1)^2}$.

Étudier le signe de $f'$ sur $[0~,~+\infty[$ et en déduire le tableau de variations de $f$.
		\item Au bout de combien de temps après l’injection la concentration de l’antibiotique est-elle
maximale ? Préciser cette concentration maximale en mg$\cdot \text{L}^{-1}$.
	\end{enumerate}
\item En antibiothérapie, on définit la CMI comme la concentration minimale d’antibiotique permettant
d’empêcher la multiplication bactérienne. La CMI de l’antibiotique injecté est égale à 2,4 mg$\cdot \text{L}^{-1}$.
	\begin{enumerate}
		\item Montrer que, pour tout réel $t$ positif ou nul, $f(t) - 2,4 =\dfrac{-2,4t^2+8t-2,4}{t^2+1}$.
		\item Étudier le signe de cette expression sur l’intervalle $[0~,~+\infty[$.
		\item Montrer que la concentration de l’antibiotique injecté est supérieure à sa CMI pendant 2 h 40.
	\end{enumerate}
\item
	\begin{enumerate}
		\item Vérifier que la fonction $F$ définie sur $[0~,~+\infty[$ par $F(t)= 4 \ln( t^2+1)$ est une primitive de $f$ sur cet intervalle.
		\item En déduire la valeur exacte de l’intégrale $\displaystyle J=\int_0^{12} f(t)\mathrm{d}t$.
		\item On admet que la valeur moyenne de la concentration de l’antibiotique en mg$\cdot\text{L}^{-1}$ durant les douze premières heures après l’injection est égale à $ \dfrac{1}{12}J$.

Déterminer cette valeur moyenne, arrondie au centième de mg$\cdot \text{L}^{-1}$.
	\end{enumerate}
\end{enumerate}

\newpage

\rotatebox{90}{
\psset{xunit=0.075cm,yunit=0.0025cm,labelFontSize=\scriptstyle,arrowsize=2pt 3}
\begin{pspicture}(-15,-700)(280,3200)
 \multido{\n=0+20}{13}{\psline[linewidth=0.75pt,linecolor=lightgray](\n,-2600)(\n,3000)}
 \multido{\n=-2600+100}{57}{\psline[linewidth=0.75pt,linecolor=lightgray](0,\n)(260,\n)}
 \psaxes[linewidth=0.95pt,Dx=20,Oy=5000,Dy=200,]{->}(0,0)(0,-2600)(260,3100)
\uput[u](120,3200){\textbf{\textsc{Exercice} 1: annexe à rendre avec la copie}}
\end{pspicture}
}
\end{document}