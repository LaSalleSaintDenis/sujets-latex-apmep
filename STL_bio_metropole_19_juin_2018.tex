\documentclass[10pt,a4paper,french]{article}
\usepackage{babel}
\usepackage[T1]{fontenc}
\usepackage[utf8]{inputenc}
\usepackage{fourier}
\usepackage{xspace}
\usepackage[scaled=0.875]{helvet}
\renewcommand{\ttdefault}{lmtt}
\usepackage{amsmath,amssymb,amstext,makeidx}
\usepackage{fancybox}
\usepackage{tabularx}
\usepackage[normalem]{ulem}
\usepackage{pifont}
\usepackage[euler]{textgreek}
\usepackage{textcomp,enumitem}
\usepackage[table]{xcolor}
\usepackage{lscape}
\usepackage{graphicx}
%tapuscrit Jean-Claude Souque
\newcommand{\euro}{\eurologo{}}
\usepackage{pst-tree,pst-plot,pst-text,pst-eucl,pst-bar,pst-func,pst-math,pstricks-add}
\usepackage[squaren, Gray, cdot]{SIunits}
\newcommand{\R}{\mathbb{R}}
\newcommand{\N}{\mathbb{N}}
\newcommand{\D}{\mathbb{D}}
\newcommand{\Z}{\mathbb{Z}}
\newcommand{\Q}{\mathbb{Q}}
\newcommand{\C}{\mathbb{C}}
\usepackage[left=3.5cm, right=3.5cm, top=3cm, bottom=3cm]{geometry}
\newcommand{\vect}[1]{\overrightarrow{\,\mathstrut#1\,}}
\renewcommand{\theenumi}{\textbf{\arabic{enumi}}}
\renewcommand{\labelenumi}{\textbf{\theenumi.}}
\renewcommand{\theenumii}{\textbf{\alph{enumii}}}
\renewcommand{\labelenumii}{\textbf{\theenumii.}}
\def\Oij{$\left(\text{O}~;~\vect{\imath},~\vect{\jmath}\right)$}
\def\Oijk{$\left(\text{O}~;~\vect{\imath},~\vect{\jmath},~\vect{k}\right)$}
\def\Ouv{$\left(\text{O}~;~\vect{u},~\vect{v}\right)$}
\usepackage{fancyhdr}
\makeatletter
\def\hlinewd#1{%
\noalign{\ifnum0=`}\fi\hrule \@height #1 %
\futurelet\reserved@a\@xhline}
\makeatother

\usepackage[dvips]{hyperref}
\hypersetup{%
pdfauthor = {APMEP},
pdfsubject = {Baccalauréat STL biotechnologies},
pdftitle = {Métropole - 19 juin 2018 },
allbordercolors = white,
pdfstartview=FitH}

\usepackage[np]{numprint}
\renewcommand\arraystretch{1.2}
\newcommand{\e}{\text{e}}
\frenchbsetup{StandardLists=true}
\begin{document}
\setlength\parindent{0mm}

\rhead{\textbf{A. P{}. M. E. P{}.}}
\lhead{\small Baccalauréat  Sciences et Technologies de laboratoire (STL)spécialité Biotechnologies}
\lfoot{\small{Métropole--La Réunion}}
\rfoot{\small{19 juin 2018}}
%\renewcommand \footrulewidth{.2pt}
\pagestyle{fancy}
\thispagestyle{empty}
\begin{center}{\Large \textbf{\decofourleft~Baccalauréat STL spécialité biotechnologies  ~\decofourright\\[5pt]Métropole--La Réunion 19 juin 2018}}
\end{center}

\vspace{0,25cm}
 
\textbf{\textsc{Exercice 1} \hfill 4 points}  

\vspace{0.25cm}


La plus ancienne méthode de conservation des aliments pratiquée par l'homme est la déshydratation.
Ce procédé consiste à utiliser une source de chaleur pour faire évaporer de l'eau d'un aliment.

Dans tout l'exercice, on s'intéresse à un abricot frais placé dans un séchoir pour le déshydrater.
Avant déshydratation, cet abricot frais a une masse de \np[g]{45}  dont 85\,\% d'eau. Le processus de
déshydratation s'achève lorsque cet abricot a une masse de \np[g ]{9} dont 25\,\% d'eau, il bénéficie alors de
l'appellation \og abricot sec \fg.
\begin{enumerate}
\item  Calculer la masse d'eau contenue dans cet abricot frais.
\item Vérifier que cet abricot ayant l'appellation \og abricot sec\fg{} ne contient plus que \np[g]{2.25}  d'eau.
\end{enumerate}

Soit $f$ la fonction qui, à toute durée $t$ exprimée en heures, associe la masse d'eau (en grammes)
contenue dans cet abricot placé dans le séchoir depuis $t$ heures. On admet que pour tout réel $t$ de
l'intervalle [0,13], $f(t) = 38,25\text{e}^{-0,26t}$. En annexe 1, on a tracé la courbe représentative $\mathcal{C}$ de la
fonction $f$.

\begin{enumerate}
\item \begin{enumerate}
\item  Calculer la masse d'eau présente dans cet abricot après deux heures passées dans le
séchoir. On arrondira à $10^{-2}$ g.
\item Si on laisse cet abricot dans le séchoir pendant 8 heures, pourra-t-il bénéficier de
l'appellation \og abricot sec\fg{} ? Justifier votre réponse.
\item Déterminer le temps de séchage nécessaire pour que l'abricot placé dans le séchoir puisse
bénéficier de l'appellation \og abricot sec \fg. On donnera le résultat à la minute près.
\end{enumerate}
\item On considère maintenant la totalité du processus de déshydratation qui permet de passer de
l'abricot frais, contenant 38,25 g d'eau, à l'abricot ayant l'appellation \og abricot sec \fg,
contenant \np[g]{2.25} d'eau.

Camille affirme : \og dans ce processus, le temps nécessaire pour éliminer les 5 derniers
grammes d'eau est environ 15 fois le temps nécessaire à l'élimination des 5 premiers
grammes d'eau! \fg.

Cette affirmation est-elle exacte? Justifier. On pourra utiliser la représentation graphique de
l'annexe 1 (dans ce cas, on rendra l'annexe 1 avec la copie et on laissera les traits de
construction nécessaires apparents).

\end{enumerate}

\vspace{0.5cm}

\textbf{\textsc{Exercice 2} \hfill 4 points}

\vspace{0.5cm}


Un fabricant a mis au point une machine permettant de fabriquer des blocs de glace (utilisables sur
les bateaux de pêche par exemple). L'épaisseur des blocs de glace fabriqués dépend du temps de
congélation.

On obtient le tableau ci-dessous :

\medskip

\begin{tabularx}{\linewidth}{|m{5.5cm}|*{7}{>{\centering \arraybackslash}X|}}
\hline 

Temps $t_i$ de
congélation(en heures)& 1& 2& 4 &8& 12 &18& 26\\\hline

Épaisseur $y_i$ de la
glace(en cm) &4 &8 &11& 16,5& 20,5& 24,5& 28,5\\\hline
\end{tabularx}

\medskip

On pose $x_i=\ln t_i$.
\begin{enumerate}
\item  Recopier et compléter le tableau ci-dessous. Les valeurs seront arrondies au dixième.

\medskip

\begin{tabularx}{\linewidth}{|m{5cm}|*{7}{>{\centering \arraybackslash}X|}}
\hline 

\centering $x_i$& & &  &&  && \\\hline

Épaisseur $y_i$ de la
glace(en cm) &4 &8 &11& 16,5& 20,5& 24,5& 28,5\\\hline
\end{tabularx}


\item Représenter le nuage de points de coordonnées ($x_i ,y_i$) dans le repère orthogonal fourni à
\textbf{l'annexe} 2, qui est \textbf{à rendre avec la copie.}
\item À l'aide de la calculatrice, donner une équation de la droite d'ajustement $(d)$ de $y$ en $x$
obtenue par la méthode des moindres carrés sous la forme $y = ax + b$ où les coefficients $a$ et
$b$ seront arrondis à $10^{-2}$.
\end{enumerate}Pour la suite, on prend comme modèle d'ajustement, la droite $(d)$ d'équation $y = 7,4x + 2,5$.
\begin{enumerate}[resume]
\item Tracer cette droite $(d)$ dans le repère de l'\textbf{annexe 2}.
\item Déterminer, selon le modèle d'ajustement pris, et à l'heure près, le temps nécessaire pour
fabriquer un bloc de glace de \np[cm]{32}  d'épaisseur.
\end{enumerate}

\vspace{0.5cm}

\textbf{\textsc{Exercice 3} \hfill 6 points}  

\vspace{0.5cm}


Une colonie de bactéries est mise en culture avec du glucose.

Pendant la 1\iere période de 10 minutes, la masse de glucose absorbé par la colonie de bactéries est
égale à 18,3 femtogrammes (1 gramme est égal à $10^{15}$ femtogrammes).

Pendant la $2\ieme$ période de 10 minutes, la masse de glucose absorbé par la colonie de bactéries
augmente de 26\,\% par rapport à la masse de glucose absorbé pendant la $1\iere$ période.
\begin{enumerate}
\item  Justifier que la masse de glucose absorbé pendant la $2\ieme$ période de 10 minutes est égale à
23,058 femtogrammes.
\end{enumerate}
Dans la suite, on étudie l'évolution de la masse de glucose absorbé par la colonie de bactéries en
prenant le modèle suivant:
\begin{itemize}
\item  pour tout entier naturel $n$ supérieur ou égal à 1, on note $u_n$ la masse, en femtogrammes, de
glucose absorbé pendant la n-ième période de 10 minutes ;
\item pour tout entier naturel $n$ supérieur ou égal à 1, la masse de glucose $u_{n+1}$ absorbé par la
colonie de bactéries pendant la ($n$+1)-ième période de 10 minutes augmente de 26\,\% par
rapport à la masse de glucose $u_n$ absorbé pendant la $n$-ième période de 10 minutes précédente.
\end{itemize}
\begin{enumerate}[resume]
\item 
\begin{enumerate}
\item Préciser les valeurs de $u_1$ et $u_2$.
\item Quelle est la nature de la suite $\left(u_n\right)$ ? Justifier votre réponse.
\item Pour tout entier naturel $n$ supérieur ou égal à 1, exprimer $u_n$ en fonction de $n$.
\item Calculer la masse de glucose absorbé pendant la $7\ieme$ période de 10 minutes. On donnera un
résultat arrondi à 0,1 femtogramme.
\end{enumerate}
\item On considère l'algorithme suivant:

\medskip
\begin{center}



\begin{tabular}[]{|l|}
\hline
$n\longleftarrow  1$\\
$u \longleftarrow  18,3$\\
Tant que $u \leqslant 100$\\
\hspace{2em}$n\longleftarrow n+1$\\
\hspace{2em}$u \longleftarrow 1,26\times u$\\
Fin Tant que\\
\hline
\end{tabular}
\end{center}

\medskip

Quelle est la valeur de la variable $n$ à la fin de l'exécution de l'algorithme ? Interpréter ce résultat
dans le contexte de l'exercice.
\end{enumerate}

Dans la suite de l'exercice, on s'intéresse à la masse totale de glucose absorbé depuis le début de la
mise en culture. Dans ce cadre, on exploite la feuille de calcul suivante obtenue à l'aide d'un
tableur:

\begin{center}
\begin{tabularx}{0.5\linewidth}{|c|*{3}{>{\centering \arraybackslash}X|}}
\hline 
& A& B&  C \\\hline
1&  $n$& $u_n$& $S_n$\\\hline
2& 1& 18,3& 18,3\\\hline
3& 2& 23,058& 41,358\\\hline
4& 3& 29,05308& 70,41108\\\hline
5& 4& 36,6068808& 107,017961\\\hline
6& 5& 46,1246698& 153,142631\\\hline
\end{tabularx}
\end{center}


\begin{enumerate}[resume]
\item  \begin{enumerate}
		\item  Interpréter la valeur de la cellule C4 dans le contexte de l'exercice.
		\item Quelle formule a été entrée dans la cellule C3 pour obtenir, par recopie vers le bas, les
		valeurs suivantes de la colonne C ?
		\end{enumerate}
\item Déterminer le nombre d'heures nécessaire, depuis le début de la mise en culture, à l'absorption
de 1 gramme de glucose par la colonie de bactéries (on rappelle que 1 gramme est égal à
$10^{15}$ femtogrammes).
\end{enumerate}

\vspace{0.5cm}

\textbf{\textsc{Exercice 4} \hfill 6 points}

\vspace{0.5cm}


\emph{Les trois parties sont indépendantes.}

\vspace{0.5cm}

Dans une ville, un cardiologue s'intéresse à la tension artérielle (systolique), mesurée en millimètres
de mercure (mmHg), des femmes de plus de 60 ans.

\medskip

\textbf{Partie A}

\medskip

On note $T$ la variable aléatoire qui, à chaque dossier médical d'une femme de la ville de plus de
60 ans, associe la tension artérielle de cette femme mesurée en mmHg. On suppose que $T$ suit la loi
normale d'espérance \textmu = 134 et d'écart-type $\sigma = 8,5$.
\begin{enumerate}
\item  Le cardiologue choisit au hasard le dossier médical d'une femme de plus de 60 ans parmi les
dossiers médicaux des femmes de la ville.
\begin{enumerate}
\item  Quelle est la probabilité que la tension artérielle de cette femme soit comprise entre
130 et \np [mmHg]{140} ? On donnera la valeur arrondie au millième.
\item Quelle est la probabilité que la tension artérielle de cette femme soit supérieure à
 \np [mmHg]{140} ? On donnera la valeur arrondie au millième.
\end{enumerate}
\item Donner un nombre entier $h$ tel que $P(134 - h \leqslant T \leqslant 134 +h)\approx 0,95$ (à $10^{-2}$ près). Interpréter
cette probabilité dans le contexte de l'exercice.
\end{enumerate}

\medskip

\textbf{Partie B}

\medskip

On admet que 24\,\% des femmes de plus de 60 ans de la ville étudiée sont atteintes d'hypertension
artérielle. On constitue au hasard un échantillon composé de 7 dossiers médicaux de femmes de plus
de 60 ans dans la ville étudiée. Le nombre total de dossiers médicaux de femmes de plus de 60 ans
dans cette ville est suffisamment élevé pour que l'on puisse assimiler ce prélèvement à un tirage avec
remise. On note $X$ la variable aléatoire qui prend pour valeurs le nombre de dossiers médicaux de
femmes atteintes d'hypertension artérielle dans un échantillon de 7 dossiers médicaux.

\begin{enumerate}
\item  Quelle est la loi suivie par $X$? En donner les paramètres.
\item  On donne ci-dessous la représentation graphique de la loi suivie par $X$ (en abscisses, on lit les
valeurs prises par $k$ et en ordonnées, les valeurs prises par $P(X = k))$ :

\medskip

\begin{center}
\psset{xunit=1.2cm,yunit=14cm, comma=true}
\begin{pspicture}(-0.5,-0.05)(9,0.35)
\multido{\n=0+0.01}{37}{\psline[linewidth=0.75pt,linecolor=lightgray](0,\n)(8,\n)}
\multido{\n=0+0.05}{8}{\psline[linewidth=1pt,linecolor=gray](0,\n)(8,\n)}
\psaxes[linewidth=1pt,Dy=0.05,labels=y]{-}(0,0)(-0.001,-0.0001)(8.2,0.36)
\uput[d](0.5,-0.01){0}\uput[d](1.5,-0.01){1}\uput[d](2.5,-0.01){2}\uput[d](3.5,-0.01){3}\uput[d](4.5,-0.01){4}\uput[d](5.5,-0.01){5}\uput[d](6.5,-0.01){6}
\uput[d](7.5,-0.01){7}\uput[l](0,0){0}
\psframe[fillstyle=solid,fillcolor=gray](0.35,0)(0.65,0.1464)
\psframe[fillstyle=solid,fillcolor=gray](1.35,0)(1.65,0.3237)
\psframe[fillstyle=solid,fillcolor=gray](2.35,0)(2.65,0.3067)
\psframe[fillstyle=solid,fillcolor=gray](3.35,0)(3.65,0.1614)
\psframe[fillstyle=solid,fillcolor=gray](4.35,0)(4.65,0.051)
\psframe[fillstyle=solid,fillcolor=gray](5.35,0)(5.65,0.0097)
\end{pspicture}

\end{center}
\begin{enumerate}
\item  À l'aide du graphique, déterminer une valeur approchée à 0,01 près de la probabilité pour
qu'il y ait au moins 4 dossiers médicaux de femmes atteintes d'hypertension artérielle
dans un échantillon de 7 dossiers médicaux. On détaillera la démarche.

\item Expliquer ce qui se passe sur la représentation graphique pour $X = 6$ et $X = 7$ .
\end{enumerate}
\end{enumerate}

\medskip

\textbf{Partie C}

\medskip

Un centre hospitalier universitaire souhaite comparer l'efficacité de deux régimes alimentaires
distincts, notés A et B, destinés à réduire l'hypertension artérielle dans la population des femmes de
plus de 60 ans de la ville.
Il constitue, au hasard, deux groupes de 200 femmes de plus de 60 ans de la ville souffrant
d'hypertension artérielle:

\begin{itemize}
\item [--] après avoir suivi le régime A, 15 femmes du premier groupe de 200 femmes n'ont pas de
réduction de leur hypertension artérielle ;
\item [--]après avoir suivi le régime B, 50 femmes du second groupe de 200 femmes n'ont pas de
réduction de leur hypertension artérielle.
\end{itemize}
En exploitant la notion d'intervalle de confiance, peut-on parler de différence significative
d'efficacité entre les deux régimes alimentaires en termes de réduction d'hypertension artérielle?


\newpage

\begin{landscape}
\begin{center}
\textbf{Annexe 1 (exercice 1): courbe représentative $\mathcal{C}$ de la fonction $f$}\\
À rendre avec la copie.
\end{center}
\psset{xunit=1.6cm,yunit=0.25cm}
\begin{pspicture}(-1.25,-5)(13.5,40)
\multido{\n=-1+0.25}{57}{\psline[linewidth=0.75pt,linecolor=lightgray](\n,-4)(\n,39)}
\multido{\n=0+1}{14}{\psline[linewidth=1pt,linecolor=gray](\n,-4)(\n,39)}
\multido{\n=-4+1}{43}{\psline[linewidth=0.75pt,linecolor=lightgray](-1,\n)(13,\n)}
\multido{\n=0+5}{8}{\psline[linewidth=1pt,linecolor=gray](-1,\n)(13,\n)}
\psaxes[linewidth=0.95pt,,Dy=5,]{->}(0,0)(-1,-4)(13.4,39)
\psplot[linewidth=1pt,linecolor=blue,plotpoints=5000]{0}{13}{ 2.71828 x 0.26 neg mul exp 38.25 mul }
\end{pspicture}
\end{landscape}

\newpage

\begin{landscape}
\begin{center}
\textbf{Annexe 2 (exercice 2):pour la représentation graphique}  \\
À rendre avec la copie.
\end{center}
\psset{xunit=5.5cm,yunit=0.3cm,comma=true}
\begin{pspicture}(-0.25,-1)(4.25,40)
\multido{\n=-0.1+0.1}{43}{\psline[linewidth=0.75pt,linecolor=lightgray](\n,-2)(\n,37)}
\multido{\n=0+1}{5}{\psline[linewidth=1pt,linecolor=gray](\n,-0.5)(\n,37)}
\multido{\n=-2+1}{40}{\psline[linewidth=0.75pt,linecolor=lightgray](-0.2,\n)(4.1,\n)}
\multido{\n=0+5}{8}{\psline[linewidth=1pt,linecolor=gray](-0.1,\n)(4.1,\n)}
\psaxes[linewidth=0.95pt,Dx=0.5,Dy=5,]{->}(0,0)(-0.1,-2)(4.2,37)
\uput[dl](0,0){0}
\end{pspicture}
\end{landscape}
\end{document}